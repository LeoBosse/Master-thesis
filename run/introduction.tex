%% ==============================
\chapter*{Introduction}
\addcontentsline{toc}{chapter}{Introduction}
\label{ch:introduction}
%% ==============================

%
%
%History of CR  and GR detection.
%	-> 20th century to today
%Why are they so important
%	-> highest particle energy, we are not able to recreate that
%	-> good understanding of their origin and evolution can inform on the sources, the galaxy, all that we havn't thought of yet (DM?)
%GC excess in GR
%	-> problem in our understanding of CR
%	-> Something is missing : DM, MSP or MCs
%Lets compare the 3 main hypothesis
%	-> fitting the latest observations with current theory
%
%Problem : Which of these hypothesis works better ?
%	
%Plan:
%	-> theory
%	-> method
%	-> results and discussion

Cosmic Ray (CR) physics started in the early 20th century with the study of the atmospheric ionization. First, highly energetic particles were believed to be linked to the Earth's natural radioactivity. The first to point out the extraterrestrial origin of the atmosphere ionization was Victor Hess in 1912, when he sent an electrometer in a balloon to high altitudes. The fact that the ionization increased with altitude was a surprise and its discovery gave him a Nobel Price. Multiple studies have then confirmed and refined the results. CR origin and composition was unknown for a decade, until the 30's where discoveries multiplied. First thought to be gamma-rays, anisotropies in their arrival direction indicated a higher proportion of positively charged particles, protons, as later balloons experiments identified. It was also during that time that particle showers were detected for the first time and showed the very high energy of CR. No experiment on Earth was able to recreate such energies to this day, so their study is complicated and often based on indirect observations. This makes these particles the only means of studying highly energetic physical phenomena, either at small scales for particle physics or very large for astrophysics, and both are strongly linked together. 

Even though their collision processes are relatively well known, their origin still presents many mysteries. Identifying CR sources precisely is a hard task and is still subject to many studies today. One of the first ideas was supernovae, and many other candidates have joined the list, such as pulsars, quasars and gamma-ray bursts. The understanding of these processes could teach a lot about every one of these powerful objects, and contribute to the understanding of the universe.

There exists a few processes for CR particles to interact in space, and these are known and can be studied on Earth. These processes are sources of indirect particle creation, particularly gamma-rays that can be used for indirect detection methods. These high energy photons are useful because they travel in a straight path, but observations and theories do not match perfectly yet. Especially near the galactic center, an excess of gamma-rays is observed that cannot be explained. The high density of matter, radiation and processes happening in that region leave many possible interpretations. And, as expected, several hypotheses on the origin of the excess already exist, some of them going beyond the standard model of particle physics.

The goal of this thesis will be to compare the three main hypotheses to explain this excess, namely dark matter annihilation, millisecond pulsars and molecular clouds cut-offs. The study will be based on data-driven methods and try to decide which theory seems most likely. The entire gamma-sky will be fitted to our model in a large energy range, enabling spectral and spatial studies. If one of the three hypotheses stand out from the others, later improvements can be made to refine the study even further.

The following chapters will first present the detailed theory behind cosmic-ray physics, relevant for the rest of the thesis. Then the scientific method used to obtain the results will be detailed. This will lead to the first results in themselves and discussions on their meaning and relevance. Finally, an examination of possible changes to the model that could improve the results will introduce the final conclusion.


	
	



