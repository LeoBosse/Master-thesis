%% ==================
\chapter{Discussion}
\label{ch:discussion}
%% ==================
%This chapter is dedicated to a discussion of the results obtained in the previous Chapter \ref{ch:results}.
%Discussions what do they mean?:


%%%FOLLOWING SECTION IS DONE IN RESULT CHAPTER AS WE PRESENT THE MCR, DM AND MSP RESULTS
%\section{Comparison of all 3 excess templates}
%	-Comparison of all 3 templates -> best is MCR from the chi2 maps
%		-Comparison MCR vs DM
%		-Comparison MCR vs MSP
%		-Weniger plots
%		-Specklings
%	

\section{Interpretation of spatial shapes}
%	-Interpretation of spatial shapes
%		-Expected features
%			-PCR			
%			-IC spherical
%			-BR a little everywhere, less in bubbles			
%			-SCR in disk + bubbles
%		-Sandwich in IC and PCR
%			-MCR replaces PCR -> There is more MCs in disk. Blocks PCR
%			-IC in disk mostly due to Dust and starlight is blocked by it. Dust is only a small portion compared to starlight -> decrease
%		
%		-Shape of the excess component		
%			-Shape of MCR, DM, MSP -> traces the CO map
%			-Only make sense with MCR
%			-DM and MSP have no reason to do that



\section{Why is MCR better than DM or MSP}
%	-Why is MCR better than DM or MSP
%		-spectral shape of DM
%			-falls off too steep
%		-spectral shape of MSP
%			-Is too soft at low energies
%

Gavering the results of the previous section, a conclusion tends to emerge : the fit clearly prefers the MCR hypothesis over DM and MSP. We will discuss why in this section.

First of all, comparing the $\chi^2$ skymaps is pretty clear. Adding a fifth component over SCR is a improvement in all cases, but the best fits arre obtained with MCR, especially in the disk. 




\section{How do these results fit in context}
%How do they fit in context:
%	-DM explanation
%		-Best mass around 50 GeV
%		-Increase goodness of fit
%		-Not spherical at all
%		-Follows CO map when there is no reason to do it
%	-MSP explanation
%		-Spectra too soft at low energies.
%		-Same problem than with DM.
%		
%	-What's new?



