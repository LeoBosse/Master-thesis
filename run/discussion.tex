%% ==================
\chapter{To go further: How to improve the model}
\label{ch:discussion}
%% ==================
%This chapter is dedicated to a discussion of the results obtained in the previous Chapter \ref{ch:results}.
%Discussions what do they mean?:


%%%FOLLOWING SECTION IS DONE IN RESULT CHAPTER AS WE PRESENT THE MCR, DM AND MSP RESULTS
%\section{Comparison of all 3 excess templates}
%	-Comparison of all 3 templates -> best is MCR from the chi2 maps
%		-Comparison MCR vs DM
%		-Comparison MCR vs MSP
%		-Weniger plots
%		-Specklings
%	

%\section{Interpretation of spatial shapes}
%	-Interpretation of spatial shapes
%		-Expected features
%			-PCR			
%			-IC spherical
%			-BR a little everywhere, less in bubbles			
%			-SCR in disk + bubbles
%		-Sandwich in IC and PCR
%			-MCR replaces PCR -> There is more MCs in disk. Blocks PCR
%			-IC in disk mostly due to Dust and starlight is blocked by it. Dust is only a small portion compared to starlight -> decrease
%		
%		-Shape of the excess component		
%			-Shape of MCR, DM, MSP -> traces the CO map
%			-Only make sense with MCR
%			-DM and MSP have no reason to do that


\section{Discussion on the spatial distributions of the other components.}

The spatial distribution of a component can teach a lot about the fit. Every component is closely linked to a known physical process taking place in the Milky way. So even if the components contributions to a single cone can not be predicted, large scale structures  should emerge and coincide with the predictions. Using this, it is possible to verify the proper functioning of the fit by comparing the fitted spatial distribution of the different templates and the predicted one.
For example, the distribution of the ISRF is expected to be spherical around the GC in the UV range, where most of the starlight is emitted. It can also follow the dust distribution in the disk in the infra-red range. This can be used to check if the IC component is coherent in the fit and follows, to some extent, the ISRF distribution. Hopefully, the results show a spherical IC component centred on the GC.
The different distributions of each component will be discussed in the following section.

\subsection{PCR flux distribution}
The PCR component is produced by diffuse CR protons which collide with other hadrons and form a neutral pion, which in turn decays into two photons. So the PCR flux is directly proportional to diffuse CR, gas and dust density in the galaxy along the line of sight. It is expected to be present in every direction, with a stronger flux coming from the disc, and less from high latitudes.
The bubbles have a harder CR proton spectrum, but composed of source protons. This could imply that more high energy pion are created in those regions, and so the PCR flux should be more important. But this is not true since the presence of the SCR component. The latter was created specifically for such regions with a hard CR spectrum and should take care of that instead of PCR. This way, PCR should not mark too much the bubbles shape.

The results are very similar for the three fits, with MCR, DM or MSP. The galactic disk is clearly visible at all longitudes, up to twenty degrees in latitude. Some structures from the gamma ray sky are also here, with the bulge around the GC and diffuse shapes at the anti-center.
Looking carefully at the disk below two degrees in latitude near the GC, the flux decreases a little, when it is supposed to be the highest. Indeed, the GC is were the density of matter along the line of sight is the most important, and so PCR should follow it. This structure could be interpreted as the result of a higher concentration of molecular clouds in this area, cutting off low energy CR, as modelled by MCR. This effect could affect the proportion of PCR gamma-rays in favour of MCR and reverse the relative contributions. Thus MCR should replace PCR, but the sum of both component should still show an increase in flux toward the GC. Indeed, the sum skymaps \todo{add picture !!!} is coherent. Like for the bubbles, the propagated proton CR spectrum changes too much to be described by a single power law. If this interpretation is correct and it is not just a fitting mistake, it is an other argument that can easily be explained by the MCR hypothesis, but not by DM or MSP.


\begin{figure}[h]
  \centering
  \begin{minipage}[h]{0.3\textwidth}
  	\centering
	\includegraphics[width=1.\linewidth]{pic/discussion/MCRonly_fine_PCR_integral_distribution.png}
  	\subcaption{MCR}
  	\label{}
  \end{minipage}
  \hfill
  \begin{minipage}[h]{0.3\textwidth}
	  \centering
	  \includegraphics[width=1.\linewidth]{pic/discussion/DMonly_fine_PCR_integral_distribution.png}
	  \subcaption{DM}
	  \label{}
  \end{minipage}
  \hfill
  \begin{minipage}[h]{0.3\textwidth}
	  \centering
	  \includegraphics[width=1.\linewidth]{pic/discussion/MSPonly_fine_PCR_integral_distribution.png}
	  \subcaption{MSP}
	  \label{}
  \end{minipage}
  \caption{PCR spatial distribution for the three different excess fits.}
  \label{fig:PCR_flux_distrib_excess_comp}	 
\end{figure}



\subsection{IC flux distribution}
The inverse compton component is directly proportional to the ISRF and the electrons CR. Electrons are present everywhere in the galaxy, and the ISRF is dominant in the GC for the UV range, but follows the dust in the IR range and is isotropic in the radio. This distribution should be visible in the IC component of the fit.
Indeed, a spherical distribution can clearly be identified for the three different fits with DM, MCR and MSP. Nonetheless, there is a noticeable difference in the disk between the IC distribution for the MCR fit and the DM and MSP fits. In the MCR fit, the disk presents a high IC flux when DM and MSP do not. On the contrary, the MSP and DM fit present a clear dip in the galactic disk. This gap is unexpected, since the ISRF is supposed to be the strongest in the disk, and the diffuse electrons are also produced mainly here. \todo{explain this}

\begin{figure}[h]
  \centering
  \begin{minipage}[h]{0.3\textwidth}
  	\centering
	\includegraphics[width=1.\linewidth]{pic/discussion/MCRonly_fine_IC_integral_distribution.png}
  	\subcaption{MCR}
  	\label{}
  \end{minipage}
  \hfill
  \begin{minipage}[h]{0.3\textwidth}
	  \centering
	  \includegraphics[width=1.\linewidth]{pic/discussion/DMonly_fine_IC_integral_distribution.png}
	  \subcaption{DM}
	  \label{}
  \end{minipage}
  \hfill
  \begin{minipage}[h]{0.3\textwidth}
	  \centering
	  \includegraphics[width=1.\linewidth]{pic/discussion/MSPonly_fine_IC_integral_distribution.png}
	  \subcaption{MSP}
	  \label{}
  \end{minipage}
  \caption{IC spatial distribution for the three different excess fits.}
  \label{fig:IC_flux_distrib_excess_comp}	 
\end{figure}


\subsection{BR flux distribution}
The BR component is linked to the CR electron density along the line of sight and the electromagnetic fields in the galaxy. It should then be present everywhere. It is mainly what can be observed here, but some features are remarkable. Mainly the decrease in flux in the disk and the bubbles.
Depending on the fit, the average flux is also changing. The BR component is a lot more present with the MCR fit than with DM, which in turn presents more BR than the MSP fit. The fact that the MSP fit does not use a lot of BR is consistent with the fact that the MSP spectrum is very soft at low energies. Since BR is also dominant at these energies, both can not have a high flux at the same time. The fit must do concession in order to use them both.

\begin{figure}[h]
  \centering
  \begin{minipage}[h]{0.3\textwidth}
  	\centering
	\includegraphics[width=1.\linewidth]{pic/discussion/MCRonly_fine_BR_integral_distribution.png}
  	\subcaption{MCR}
  	\label{}
  \end{minipage}
  \hfill
  \begin{minipage}[h]{0.3\textwidth}
	  \centering
	  \includegraphics[width=1.\linewidth]{pic/discussion/DMonly_fine_BR_integral_distribution.png}
	  \subcaption{DM}
	  \label{}
  \end{minipage}
  \hfill
  \begin{minipage}[h]{0.3\textwidth}
	  \centering
	  \includegraphics[width=1.\linewidth]{pic/discussion/MSPonly_fine_BR_integral_distribution.png}
	  \subcaption{MSP}
	  \label{}
  \end{minipage}
  \caption{BR spatial distribution for the three different excess fits}
  \label{fig:BR_flux_distrib_excess_comp}	 
\end{figure}

\subsection{SCR flux distribution}
The SCR distribution is expected to follow the disc, where point sources can still remain, and the bubbles, where the proton CR spectra is harder. And indeed, the spatial distribution obtained in the three  different fits are similar and correspond perfectly to the expectations, showing a great coherence between the model and the theory, but not very helpful to study the excess.


\begin{figure}[h]
  \centering
  \begin{minipage}[h]{0.3\textwidth}
  	\centering
	\includegraphics[width=1.\linewidth]{pic/discussion/MCRonly_fine_SCR_integral_distribution.png}
  	\subcaption{MCR}
  	\label{}
  \end{minipage}
  \hfill
  \begin{minipage}[h]{0.3\textwidth}
	  \centering
	  \includegraphics[width=1.\linewidth]{pic/discussion/DMonly_fine_SCR_integral_distribution.png}
	  \subcaption{DM}
	  \label{}
  \end{minipage}
  \hfill
  \begin{minipage}[h]{0.3\textwidth}
	  \centering
	  \includegraphics[width=1.\linewidth]{pic/discussion/MSPonly_fine_SCR_integral_distribution.png}
	  \subcaption{MSP}
	  \label{}
  \end{minipage}
  \caption{SCR spatial distribution for the three different excess fits.}
  \label{fig:SCR_flux_distrib_excess_comp}	 
\end{figure}







\section{Adding the MBR component}

\subsection{MBR fixed to MCR, magnetic cut-off}
\todo{add pictures of graph and skymaps}

The theory behind the MCR component involves a magnetic cut-off for low rigidity CR protons, that changes the gamma-ray spectrum at low energies. Such a cut-off only depends on the rigidity of the CR and the intensity of the electromagnetic field. So there is no reason for it to be only applied to protons, and not to electrons. This is the reasoning that led to the introduction of a new template, called MBR in the fit, equivalent to the BR template, but calculated from a different electron CR spectrum. This new CR spectrum presents two parts, with two different power laws for high and low energies. It has the exact same spectral indexes than the spectrum for IC and BR, but the break is moved up, between 6 and 14 GeV. Then, the break is taken to have the same value than the MCR brake. This way, the magnetic cut-off is applied the same way to the electrons and protons CR. Once again, this component is expected to follow the MCs distribution, or at least to be only present in the disk.

A MIC component was tested as well, but it was not kept for physical reasons. First, the ISRF is not dependent of the electromagnetic field, photons do not carry any electric charge. It is also present in MC, where stars are formed and evolve.

\subsection{MBR free, Alfen waves}





\section{Add DM late}

\todo{difference with DM+MCR. Results and discussion.}

Now the results are clear. MCR is a good solution to the gamma-ray GeV excess in the GC, and could replace the first hypothesis of the DM halo or hidden MSPs. Yet, the DM theory is strongly supported by other measurements \todo{cite}, for example the rotation curves of galaxies, and is one of the most important question of modern physics. And the fact that it is not the primary cause of the effect observed here does not mean it does not exist. On the contrary, under the hypothesis that MCR is really the cause of the excess, the fit could put strong limits on the DM models.

This section will present a few ways to study the current DM WIMP model using the fit method of this thesis.

\subsection{DM and MCR}

\todo{add figure}
A first idea is to let the fit choose for every cone the template that offers the best $\chi^2$ value. Taking a 52.3 GeV WIMP mass and the MCR template with free breaks, the fit has the choice between the two. The results are without much surprise in favor of MCR. Indeed, the previous comparisons of the $\chi^2$ distribution for a MCR or a DM fit showed a clear preference toward MCR. The $\chi^2$ map of a MCR fit is flat around one, even in the disk, whereas a DM fit does not work well for small latitudes. It is then only expected that the fit chooses MCR when it has the choice.

\subsection{DMlate}

This method can not say much else than what was already showed in the previous sections. The second idea that comes to mind is to take the MCR template for granted, and see if their is space to add DM contribution at the end of the fit. In other words, a first fit is performed with the classical templates (PCR, IC, BR), the source (SCR) and the MCR templates. Once the best relative contributions are found, the fit tries to add a DM template to improve the $\chi^2$.
Following the Ockham's razor principle, the least amount of contributions are accepted as true, and any additional component must be treated carefully. This way, the necessity of a DM template is checked, and it contribution can not be mistaken for a classical one.
\todo{add pic}
Such a fit was performed and the results will be discussed in this section. First of all, the DM flux distribution  is very sparse and degrees of magnitude below the other components. This was expected since the $\chi^2$ of the MCR only fit was already good. Only very small changes could improve it without changing the classical contributions. Using the same CMZ region in the GC, the optimal mass was determined to be \todo{mass DM late}.

Further studies of this kind could help to determine limits on the DM particle mass and cross-section.


\section{How do these results fit in context}
%How do they fit in context:
%	-DM explanation
%		-Best mass around 50 GeV
%		-Increase goodness of fit
%		-Not spherical at all
%		-Follows CO map when there is no reason to do it
%	-MSP explanation
%		-Spectra too soft at low energies.
%		-Same problem than with DM.
%		
%	-What's new?






