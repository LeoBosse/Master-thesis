%% ==================
\chapter{Conclusion}
\label{ch:conclusion}
%% ==================

%
%Conclusion
%
%Resume what I did
%	-> fit
%	-> method
%
%Results
%	-> MCR is the fit winner
%	-> DM and MSP also work, but do not follow their predictions
%
%Opening
%	-> Put limit on DM mass and cross-section
%	-> Find other ways to study DM
%

Using a simple, well constrained, fitting method, the entire gamma-sky was modeled in a wide energy range. This allowed a spatial as well as a spectral study to be developed around the GC gamma-ray excess. High spatial resolution up to one by one degree bins were achieved, allowing a very detailed study of the excess distribution.

Three main hypothesis for the GC gamma-ray excess were tested and compared, namely the dark matter (DM), milli-second pulsars (MSP) and molecular clouds cut-offs (MCR). The results makes the MCR hypothesis stand out from DM and MSP. First, the spectral shape of the MCR component is more adapted to the excess spectral shape, making it easier for the fit to use. High energies as well as low energies can be fitted properly for any given region of the sky. On the contrary, MSP and DM have spectral features that do not allow a proper fit everywhere, especially the very soft spectrum above 10 GeV, making them insignificant for higher energies. Second, comes the general spatial distribution of the excess component in the fits. All three looks alike, following the disk and galaxy general shape, but only MCR is supposed to do so. Indeed, DM and MSP are expected to be distributed spherically around the GC, and they have no reasons to follow the galactic matter distribution. And finally, when looking at the details of the GC, and in particular the CO distribution, one can see a correlation between the excess component flux and the CO emission. This last result gives again a clear indication in favor of MCR, since it is the only component directly linked with molecular clouds. Overall, MCR was preferred by the fit and gives a explanation for the GC excess while staying in the standard model of physics.

Several path were followed to try and improve the model, taking MCR as the excess component. A new component (MBR) was introduce, following the same reasoning than MCR, but with the electron CR. Since their proportion is smaller, the effect of this new template is expected to be smaller.

