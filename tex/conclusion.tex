%% ==================
\chapter{Conclusion}
\label{ch:conclusion}
%% ==================

%
%Conclusion
%
%Resume what I did
%	-> fit
%	-> method
%
%Results
%	-> MCR is the fit winner
%	-> DM and MSP also work, but do not follow their predictions
%
%Opening
%	-> Put limit on DM mass and cross-section
%	-> Find other ways to study DM
%

Using a simple, well constrained, fitting method, the entire gamma-sky was modeled in a wide energy range. This allowed a spatial as well as a spectral study to be developed around the GC gamma-ray excess. High spatial resolution up to one by one degree bins were achieved, allowing a very detailed study of the excess distribution.

Starting from the adopted model composed of three components, PCR, BR and IC, improvements were made to solve its two main issues. First, the addition of a source template (SCR) improved the fit at high energies in the disk and the bubbles (see section \ref{sec:introducing_SCR} or figure \ref{fig:SCRonly_BKGonly_spec_comp}). Indeed, a harder spectrum was needed above 10 GeV, and a harder CR proton spectrum worked perfectly. The flux distribution from this new component follows exactly its expected distribution.

Then, three hypothesis for the GC gamma-ray excess were tested and compared, namely the dark matter (DM), milli-second pulsars (MSP) and molecular clouds cut-offs (MCR). The results makes the MCR hypothesis stand out from DM and MSP. First, the spectral shape of the MCR component is more adapted to the excess spectral shape, thus allowing high energies as well as low energies to be fitted properly for any given region of the sky. On the contrary, MSP and DM have spectral features that do not allow a proper fit everywhere, especially the very soft spectrum above 10 GeV, making them insignificant for higher energies. Second, comes the general spatial distribution of the excess component in the fits. All three looks alike, following the MCs distribution in the galaxy when only MCR is supposed to do so. Indeed, DM and MSP are expected to be distributed spherically around the GC, and they have no reasons to follow the galactic matter distribution. And finally, when looking at the details of the GC, and in particular the CO distribution, one can see a correlation between the excess component flux and the CO emission (see section \ref{sec:CO_MCR_correlation} and figure \ref{fig:CO_MCR_correlation}). This last result gives again a clear indication in favor of MCR, since it is the only component directly linked with molecular clouds. Overall, MCR was preferred by the fit and gives a explanation for the GC excess while staying in the standard model of physics.

Several path were followed to try and improve the model where MCR is the excess component. A new component (MBR) was introduce, following the same reasoning than MCR, but with the electron CR. Since electron proportion is smaller, the effect of this new template is expected to be smaller. The quality of the fit does not change significantly since it was already good allowing only for relatively small changes.
Accepting the MCR hypothesis, one can also look for DM using this fitting method. This scenario was tested but did not give significant results, as MCR is always a better choice for the fit. DM is almost not contributing to the gamma-ray flux in the galaxy, and no spherical distribution could be observed. This does not disprove the existence of DM as WIMPs, but could help to study it. In latter work, an upper limit could be found on the WIMPs density and cross section for example.
The model developed in this work can still be improved, and strange structure in some component spatial distribution need further discussion.

The main result to take away from this work is that molecular clouds could explain the presence of the Fermi GeV excess around the GC and give a good model for the diffuse gamma-ray emission for the entire galaxy.

