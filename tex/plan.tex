Plan:

Intoduction:

Physic of cosmic and gamma rays:
	-creation
		-supernovae (SNR)
		-extra-galactic
		-pulsars
	-Propagation of CR
	-Creation of gamma rays
		-pion decay
		-bremsstrahlung
		-inverse compton
		-?synchrotron radiation?
	-Gamma ray observations
		
			
	-What are the unresolved problems of the precedent chapter:
		-Bad fits in bubbles and disk
		-Spherical gamma-ray excess in GC when fitting spatial templates
		-Ideas to fill the excess			
			-DM studies
				-Hooper
				-others...
			-MSP studies
				-Fermi
				-Hooper
				-Weniger

		(-High energy tail flux too hard)

	
	
My method:
	-Spectral templates fitted to the data
		-independant spatial cones on the entire sky (usually 797 for optimized sizes)
		-minimum chi2 fit using ROOT for every cone
		-Benefits
			-energy related features
			-only a few degrees of freedom -> Well constrained fit (5(or 6) dof against 21-30 points)
		-Downside:
			-No spatial templates. (only the isosky)
			
	
	-Data origin
		-Fermi collaboration
		-LAT
		-Fermi FTOOLS
		-point source subtraction
		-pass7 vs pass8
			-wider energy range
			-better statistical and systematic errors
			-better event selection
		

	-3 Basic components
		-PCR
			-proton CR follow power-law E^-2.849
			?-try with break at 5-10GeV with index ~ -2.7 ?
		-IC and BR
			-electron CR spectrum E^-3.21
			-break at 1GeV, index E^-3.21 + 2.4 below break
		
	
	-2 or more additional components:
		-SCR
			-proton CR follow power-law E^-2.1
		-MCR
			-proton CR follow power-law E^-2.849
			-break between 6 and 14GeV, index E^-2.849 + 2.149 below break
		-DM
			-Dark Susy
			-Determination of Mass using best fit in CMZ
		-MBR
			-electron CR spectrum E^-3.21
			-break between 6 and 14 GeV, index E^-3.21 + 2.4 below break
	-Isosky
		-Calculated from fermi model and adjusted in our fit

%	-Weniger plots
%		-study of spectra slope between 0.3 and 2 GeV
%	-Specklings
%		-Study of symmetry
%	-Comparison with CO map



My results:
	-Only 3 original components
		-Hardly see any spherical excess
		-very bad chi2 in disk and bubbles

????	-Introduction of SCR
	
	
	-Introduction of SCR and MCR
		-very good chi2 in disk and bubbles
		-spatial shapes of comps
			-IC sperical (as expected) but depletion in disk
			-BR low in bubbles replaces IC in disk
			-PCR OK but low in disk
			-MCR follows CO map, take place of PCR in disk
			-SCR follows bubble structure
		-Weniger plots
		-Specklings

	-Introduction of SCR + DM
		-chi2
		-spatial shapes	
		-Weniger plots
		-Specklings
		
		-Discussion and comparison with MCR

	-Introduction of SCR + MSP:
		-chi2
		-spatial shapes
		-Weniger plots
		-Specklings
		
		-Discussion and comparison with MCR and DM

Discussions what do they mean?:
	-Comparison of all 3 templates -> best is MCR from the chi2 maps
		-Comparison MCR vs DM
		-Comparison MCR vs MSP
		-Weniger plots
		-Specklings
	
	-Interpretation of spatial shapes
		-Expected features
			-PCR			
			-IC spherical
			-BR a little everywhere, less in bubbles			
			-SCR in disk + bubbles
		-Sandwich in IC and PCR
			-MCR replaces PCR -> There is more MCs in disk. Blocks PCR
			-IC in disk mostly due to Dust and starlight is blocked by it. Dust is only a small portion compared to starlight -> decrease
		
		-Shape of the excess component		
			-Shape of MCR, DM, MSP -> traces the CO map
			-Only make sense with MCR
			-DM and MSP have no reason to do that

	
	-Why is MCR better than DM or MSP
		-spectral shape of DM
			-falls off too steep
		-spectral shape of MSP
			-Is too soft at low energies


How do they fit in context:
	-DM explanation
		-Best mass around 50 GeV
		-Increase goodness of fit
		-Not spherical at all
		-Follows CO map when there is no reason to do it
	-MSP explanation
		-Spectra too soft at low energies.
		-Same problem than with DM.
		
	-What's new?
		

Conclusion

